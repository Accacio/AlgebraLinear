%Este trabalho está licenciado sob a Licença Creative Commons Atribuição-CompartilhaIgual 3.0 Não Adaptada. Para ver uma cópia desta licença, visite https://creativecommons.org/licenses/by-sa/3.0/ ou envie uma carta para Creative Commons, PO Box 1866, Mountain View, CA 94042, USA.

%\documentclass[../livro.tex]{subfiles}  %%DM%%Escolher document class and options article, etc

%define o diretório principal
\providecommand{\dir}{..}

%%%%%%%%%%%%%%%%%%%%%%%%%%%%%%%%%%%%%%%%%%%%%
%%%%%%%%%%%%INICIO DO DOCUMENTO%%%%%%%%%%%%%%
%%%%%%%%%%%%%%%%%%%%%%%%%%%%%%%%%%%%%%%%%%%%%

%\begin{document}

\chapter{Semana 2}

\vspace{0.5cm}

Existem maneiras diferentes de representar e de interpretar elementos de Álgebra Linear. Nesta parte, vemos algumas das principais.


\section{Vetores}

De um ponto de vista intuitivo, um \textbf{vetor} no espaço tridimensional é um objeto com magnitude, direção e sentido bem definidos. Assim sendo, ao representar um vetor como na figura abaixo, não importa onde está o seu ponto inicial. Todas as setas representam o mesmo vetor.


\begin{figure}[h!]
	\begin{center}
		\includegraphics[width=0.3\linewidth]{Semana02/semana02-vetor}
	\end{center}
\end{figure}

Mais precisamente, % pode-se dizer que um \textbf{vetor} no espaço tridimensional
vamos dizer que segmentos de reta orientados  são equivalentes se tem mesmo comprimento, direção e sentido. Um vetor é a {\it classe de equivalência} de todos os segmentos de reta com mesmo comprimento, direção e sentido, ou seja, de maneira simplificada, o conjunto de todos os vetores com estas mesmas três características.
%Não vamos falar aqui em relações de equivalência

Do ponto de vista algébrico, um vetor é um elemento de um {\it espaço vetorial}. Esta definição por enquanto não faz sentido visto que somente iremos definir espaços vetoriais no futuro, mas podemos nos contentar com uma definição simplificada neste momento: do ponto de vista algébrico um vetor no espaço tridimensional pode ser pensado como um trio ordenado de números reais. A noção de base de um espaço vetorial, a ser introduzida no futuro, juntamente com a noção de coordenadas de um espaço vetorial, dará um significado preciso a esta ultima afirmação. Aqui podemos considerar a {\it base canônica}:

\begin{equation}
\vec{e_1} =
\left[
\begin{array}{c}
1 \\
0 \\
0
\end{array}
\right], \quad
\vec{e_2} =
\left[
\begin{array}{c}
0 \\
1 \\
0
\end{array}
\right] \quad  \text{e} \quad
\vec{e_3} =
\left[
\begin{array}{c}
0 \\
0 \\
1
\end{array}
\right].
\end{equation}


Fixando o sistema de coordenadas Cartesiano usual e usando a origem como referência, podemos introduzir coordenadas:
\begin{equation}
\vec{v} =
\left[
  \begin{array}{c}
    v_1 \\
    v_2 \\
    v_3
  \end{array}
\right]
= v_1 \vec{e_1} + v_2 \vec{e_2} + v_3 \vec{e_3}.
\end{equation}



\noindent Intuitivamente, podemos pensar que o sistema de coordenadas Cartesiano fixa a origem como sendo o ponto inicial do vetor e o ponto $(v_1, v_2, v_3)$ do espaço tridimensional como o ponto final.

\begin{figure}[h!]
\begin{center}
\includegraphics[width=0.2\linewidth]{Semana02/semana02-coord}
\end{center}
\end{figure}

No livro ``Álgebra Linear e suas Aplicações'', de David Lay, vetores são representados por letras em negrito. Nestas notas, utilizamos uma setinha em cima da letra para indicar um vetor.

Raramente vamos confundir o ponto $(v_1, v_2, v_3)$ com o vetor
\begin{equation}
\left[
\begin{array}{c}
v_1 \\
v_2 \\
v_3
\end{array}
\right].
\end{equation} De qualquer maneira, é bom estar atento, pois outros textos podem fazê-lo.

O conjunto de todos os vetores com três componentes como acima é chamado de $\mathbb{R}^3$ (leia-se ``erre três''). Em coordenadas, se tivermos
\begin{equation}
\vec{v} =
\left[
  \begin{array}{c}
    v_1 \\
    v_2 \\
    v_3
  \end{array}
\right]  \quad  \text{e} \quad
\vec{w} =
\left[
  \begin{array}{c}
    w_1 \\
    w_2 \\
    w_3
  \end{array}
\right],
\end{equation} podemos definir a \textbf{soma de vetores} componente a componente:
\begin{equation}
\vec{v} + \vec{w} :=
\left[
  \begin{array}{c}
    v_1 + w_1 \\
    v_2 + w_2 \\
    v_3 + w_3
  \end{array}
\right].
\end{equation} Geometricamente, isto pode ser interpretado como segue:
\begin{figure}[h!]
\begin{center}
\includegraphics[width=0.3\linewidth]{Semana02/semana02-soma}
\end{center}
\end{figure}

A \textbf{multiplicação} do vetor $\vec{v}$ \textbf{por escalar} $k \in \mathbb{R}$ é definida como:
\begin{equation}
k \vec{v} :=
\left[
  \begin{array}{c}
    k v_1 \\
    k v_2 \\
    k v_3
  \end{array}
\right].
\end{equation}


Geometricamente, temos a seguinte situação
\begin{figure}[h!]
\begin{center}
\includegraphics[width=0.3\linewidth]{Semana02/semana02-escalar}
\end{center}
\end{figure}



Estas considerações que fizemos também são válidas para outro número de componentes, com a possível perda de visualização geométrica, no caso de quatro ou mais componentes.

Mais geralmente, um vetor $\vec{v}$ do conjunto $\mathbb{R}^n$ pode ser pensado como um objeto com $n$ componentes:
\begin{equation}
\vec{v} =
\left[
  \begin{array}{c}
    v_1 \\
    v_2 \\
   \vdots \\
    v_{n-1} \\
    v_n
  \end{array}
\right]
= v_1 \vec{e}_1 + v_2 \vec{e}_2 + \cdots + v_{n-1} \vec{e}_{n-1} + v_{n} \vec{e}_{n},
\end{equation} onde
\begin{equation}
\vec{e}_1 =
\left[
  \begin{array}{c}
    1 \\
    0 \\
  \vdots \\
    0 \\
    0
  \end{array}
\right], \quad
\vec{e}_2 =
\left[
  \begin{array}{c}
    0 \\
    1 \\
    0 \\
  \vdots \\
    0
  \end{array}
\right],   \cdots, \quad
\vec{e}_{n-1} =
\left[
  \begin{array}{c}
    0 \\
    0 \\
  \vdots \\
    1 \\
    0
  \end{array}
\right], \quad
\vec{e}_n =
\left[
  \begin{array}{c}
    0 \\
    0 \\
  \vdots \\
    0 \\
    1
  \end{array}
\right].
\end{equation} A soma de vetores é dada em coordenadas por:
\begin{equation}
\vec{v} + \vec{w} :=
\left[
  \begin{array}{c}
    v_1 + w_1 \\
    v_2 + w_2 \\
    \vdots \\
    v_{n-1} + w_{n-1} \\
    v_n + w_n
  \end{array}
\right].
\end{equation} A multiplicação por escalar por:
\begin{equation}
k \vec{v} :=
\left[
  \begin{array}{c}
    k v_1 \\
    k v_2 \\
    \vdots \\
    k v_{n-1} \\
    k v_n
  \end{array}
\right].
\end{equation}








\section{Combinações lineares de vetores}

Uma \textbf{combinação linear} dos vetores $\vec{v}_1, \vec{v}_2, \dots, \vec{v}_k$ é um vetor da forma:
\begin{equation}
\vec{v} = x_1 \vec{v}_1 + x_2 \vec{v}_2 + \cdots + x_k \vec{v}_k = \sum_{i=1}^k x_i \vec{v}_i,
\end{equation} onde $x_1, x_2, \dots, x_k$ são números reais (chamados coeficientes da combinação linear). Em outras palavras, uma combinação linear é uma soma de múltiplos dos vetores $\vec{v}_1, \vec{v}_2, \dots, \vec{v}_k$.


\begin{ex}
	Para os vetores
	\begin{equation}
	\vec{v}_1 = \left[
	\begin{array}{c}
	1 \\
	-1
	\end{array}
	\right] \quad \text{e} \quad
	\vec{v}_2 = \left[
	\begin{array}{c}
	1 \\
	3
	\end{array}
	\right],
	\end{equation} algumas combinações lineares são:
	\begin{itemize}
		\item $\vec{v}_1 + \vec{v}_2 =
		\left[
		\begin{array}{c}
		1 \\
		-1
		\end{array}
		\right] +
		\left[
		\begin{array}{c}
		1 \\
		3
		\end{array}
		\right] =
		\left[
		\begin{array}{c}
		2 \\
		2
		\end{array}
		\right];$
		\item $ 4 \vec{v}_1 = 4 \vec{v}_1 + 0 \vec{v}_2 =
		4 \left[
		\begin{array}{c}
		1 \\
		-1
		\end{array}
		\right] =
		\left[
		\begin{array}{c}
		4 \\
		-4
		\end{array}
		\right];$
		\item $ \vec{v}_2 = 0\vec{v}_1 + 1\vec{v}_2 =
		\left[
		\begin{array}{c}
		1 \\
		3
		\end{array}
		\right];$
		\item $0\vec{v}_1 + 0 \vec{v}_2 =
		\left[
		\begin{array}{c}
		0 \\
		0
		\end{array}
		\right] = \vec{0}.$
	\end{itemize} \textbf{Geometricamente}, qualquer vetor do plano pode ser representado como combinação linear de vetores que não são colineares.

	Vamos ilustrar este fato na figura abaixo.
	\begin{figure}[h!]
		\begin{center}
			\includegraphics[width=0.4\linewidth]{Semana02/semana02-comblinear1}
			\includegraphics[width=0.4\linewidth]{Semana02/semana02-comblinear3}
		\end{center}
	\end{figure}
	Por exemplo, para representar o vetor $\vec{v}$ como combinação linear de $\vec{v}_1$ e $\vec{v}_2$, podemos traçar retas paralelas aos vetores, passando pela origem e pela extremidade do vetor $\vec{v}$, como na figura da direita. Assim, pela interpretação geométrica da soma de vetores (ver início deste capítulo), sabemos que $\vec{v}$ é a soma dos vetores em azul. Mas estes, por construção, são colineares aos vetores iniciais e logo múltiplos destes. Concluímos que
	\begin{equation}
	\vec{v} = \alpha \vec{v}_1 + \beta \vec{v}_2,
	\end{equation} isto é, $\vec{v}$ é uma combinação linear de $\vec{v}_1$ e $\vec{v}_2$. $\ \lhd$
\end{ex}



De forma mais geral, nós dizemos que um vetor $\vec{v} \in \mathbb{R}^m$ é combinação linear dos $k$ vetores $\vec{v}_1, \vec{v}_2, \dots, \vec{v}_k  \in \mathbb{R}^m$ quando existem coeficientes $x_1, x_2, \dots, x_k$ tais que
\begin{equation}
x_1 \vec{v}_1 + x_2 \vec{v}_2 + \dots + x_k \vec{v}_k = \vec{v}.
\end{equation} Nós vamos nos referir a este tipo de equação por \textbf{equação vetorial}. Para decidir se um vetor é combinação linear de outros, devemos decidir se existem estes números reais $x_1, x_2, \dots, x_k$. Escrevendo em coordenadas:
\begin{equation}
\vec{v}_1 =
\left[
\begin{array}{c}
v_{11} \\
v_{21} \\
v_{31} \\
\vdots \\
v_{m1}
\end{array}
\right], \
\vec{v}_2 =
\left[
\begin{array}{c}
v_{12} \\
v_{22} \\
v_{32} \\
\vdots \\
v_{m2}
\end{array}
\right], \
\vec{v}_3 =
\left[
\begin{array}{c}
v_{13} \\
v_{23} \\
v_{33} \\
\vdots \\
v_{m3}
\end{array}
\right], \ \cdots, \
\vec{v}_k =
\left[
\begin{array}{c}
v_{1k} \\
v_{2k} \\
v_{3k} \\
\vdots \\
v_{mk}
\end{array}
\right], \
\vec{v} =
\left[
\begin{array}{c}
b_{1} \\
b_{2} \\
b_{3} \\
\vdots \\
b_{m}
\end{array}
\right],
\end{equation} vemos que encontrar os coeficientes da combinação linear, caso estes existam, equivale a resolver a equação vetorial
\begin{equation}
x_1 \left[
\begin{array}{c}
v_{11} \\
v_{21} \\
v_{31} \\
\vdots \\
v_{m1}
\end{array}
\right] + x_2
\left[
\begin{array}{c}
v_{12} \\
v_{22} \\
v_{32} \\
\vdots \\
v_{m2}
\end{array}
\right] + x_3
\left[
\begin{array}{c}
v_{13} \\
v_{23} \\
v_{33} \\
\vdots \\
v_{m3}
\end{array}
\right] + \cdots + x_k
\left[
\begin{array}{c}
v_{1k} \\
v_{2k} \\
v_{3k} \\
\vdots \\
v_{mk}
\end{array}
\right] =
\left[
\begin{array}{c}
b_{1} \\
b_{2} \\
b_{3} \\
\vdots \\
b_{m}
\end{array}
\right],
\end{equation} que, por sua vez, é equivalente a
\begin{equation}
\left[
\begin{array}{c}
v_{11} x_1 + v_{12} x_2 + v_{13} x_3 + \cdots + v_{1k} x_k  \\
v_{21} x_1 + v_{22} x_2 + v_{23} x_3 + \cdots + v_{2k} x_k  \\
v_{31} x_1 + v_{32} x_2 + v_{33} x_3 + \cdots + v_{3k} x_k  \\
\vdots  \\
v_{m1} x_1 + v_{m2} x_2 + v_{m3} x_3 + \cdots + v_{mk} x_k
\end{array}
\right] =
\left[
\begin{array}{c}
b_{1} \\
b_{2} \\
b_{3} \\
\vdots \\
b_{m}
\end{array}
\right].
\end{equation} Mas isto é resolver um sistema linear! Portanto este conceito (aparentemente) novo está intimamente relacionado com tudo o que já tínhamos visto antes.


O \textbf{espaço gerado} por todas as combinações lineares dos vetores $\vec{v}_1, \vec{v}_2, \dots, \vec{v}_k$ é denotado por $\Span \{ \vec{v}_1, \vec{v}_2, \dots, \vec{v}_k \}$.

\begin{itemize}
	\item Vimos que o conjunto gerado pelos dois vetores do exemplo anterior é o plano inteiro, isto é, $\Span \{ \vec{v}_1, \vec{v}_2 \} = \mathbb{R}^2$.
	\item Nota que o vetor nulo sempre pertence ao espaço gerado, assim como todos os vetores $\vec{v}_1, \vec{v}_2, \dots, \vec{v}_k$.
	\item O conjunto gerado por apenas um vetor não nulo representa uma reta.
\end{itemize}



\section{Multiplicação de matriz por vetor e representação matricial para sistemas lineares}

Se $A$ é uma matriz com $m$ linhas e $n$ colunas, e $\vec{v}$ é um vetor do $\mathbb{R}^n$, então definimos o {\bf produto da matriz $A$ pelo vetor $\vec{v}$ } como o vetor de $\mathbb{R}^m$ resultante da combinação linear das colunas de $A$ com coeficientes dados pelas entradas do vetor $\vec{v}$.

\begin{ex}
	\begin{equation}
	\left[
	\begin{array}{ccc}
	2 &  1 & 3  \\
	4 & -3 & 5
	\end{array}
	\right]
	\left[
	\begin{array}{c}
	2   \\
	3  \\
	5
	\end{array}
	\right] = 2
	\left[
	\begin{array}{c}
    2   \\
	4
	\end{array}
	\right]
	+3 \left[
	\begin{array}{c}
	1   \\
	-3
	\end{array}
	\right]+
	5
	\left[
	\begin{array}{c}
	3   \\
	5
	\end{array}
	\right]=
	\left[
	\begin{array}{c}
	22   \\
	24
	\end{array}
	\right].
	\end{equation}
\end{ex}


Na linguagem da seção anterior, nós dizemos que dois vetores são iguais quando todas as suas componentes são iguais. Desta forma, podemos interpretar as equações de um sistema linear
\begin{equation}
  \left\{
    \begin{array}{rcl}
      x+3y&=&-1 \\
      2x-y&=&2
    \end{array}
  \right.
\end{equation} como uma igualdade entre vetores de $\mathbb{R}^2$, isto é, de vetores com duas componentes:
\begin{equation}
  \left[
    \begin{array}{c}
      x+3y \\
      2x-y
    \end{array}
  \right] =
    \left[
    \begin{array}{c}
      -1 \\
      2
    \end{array}
  \right].
\end{equation} Além disso, é desta forma que surge naturalmente o produto de uma matriz por um vetor:
\begin{equation}
  \left[
    \begin{array}{cc}
      1 & 3 \\
      2 & -1
    \end{array}
  \right]
  \left[
    \begin{array}{c}
      x \\
      y
    \end{array}
  \right] =
    \left[
    \begin{array}{c}
      -1 \\
      2
    \end{array}
  \right],
\end{equation} definido de modo que as duas últimas igualdades acima signifiquem a mesma coisa. Esta última notação é conhecida como a \textbf{forma matricial} do sistema linear.




\begin{ex}\label{example1}
O sistema, que já apareceu nas notas da semana anterior
\begin{equation}
\left\{
  \begin{array}{ll}
    x_1 + 2x_2 + x_3 = 12 \\
    x_1 -3x_2 + 5x_3 = 1 \\
    2x_1 - x_2 + 3x_3 = 10
  \end{array}
\right.
\end{equation} pode ser representado matricialmente por
\begin{equation}
\left[
  \begin{array}{ccc}
    1 &  2 & 1  \\
    1 & -3 & 5  \\
    2 & -1 & 3
  \end{array}
\right]
\left[
  \begin{array}{c}
    x_1   \\
    x_2  \\
    x_3
  \end{array}
\right] =
\left[
  \begin{array}{c}
    12   \\
    1  \\
    10
  \end{array}
\right].
\end{equation} De forma mais sucinta,
\begin{equation}
\boxed{A \vec{x} = \vec{b}}
\end{equation} onde
\begin{equation}
A = \left[
  \begin{array}{ccc}
    1 &  2 & 1  \\
    1 & -3 & 5  \\
    2 & -1 & 3
  \end{array}
\right], \quad
\vec{x} = \left[
  \begin{array}{c}
    x_1   \\
    x_2  \\
    x_3
  \end{array}
\right], \ \text{ e } \
\vec{b} = \left[
  \begin{array}{c}
    12   \\
    1  \\
    10
  \end{array}
\right].
\end{equation} Chegamos a esta conclusão colocando os coeficientes da primeira variável $x_1$ na primeira coluna, os coeficientes da segunda variável $x_2$ na segunda coluna e os coeficientes da terceira variável $x_3$ na terceira coluna$. \ \lhd$
\end{ex}


Mais geralmente, uma matriz do tipo $m\times n$ (leia-se ``$m$ por $n$'') é uma matriz com $m$ linhas e $n$ colunas:
\begin{equation}
A = \left(a_{ij}\right) =
\begin{bmatrix}
a_{11}&a_{12}&\cdots &a_{1n}\\
a_{21}&a_{22}&\cdots &a_{2n}\\
\vdots &\vdots &\ddots &\vdots \\
a_{m1}&a_{m2}&\cdots &a_{mn}
\end{bmatrix}
\end{equation} e pode estar associada a um sistema com $m$ equações e $n$ variáveis, já que o produto
\begin{equation}
\left[
  \begin{array}{cccc}
  a_{11}&a_{12}&\cdots &a_{1n}\\
  a_{21}&a_{22}&\cdots &a_{2n}\\
  \vdots &\vdots &\ddots &\vdots \\
  a_{m1}&a_{m2}&\cdots &a_{mn}
\end{array}
\right]
\left[
  \begin{array}{c}
    x_1 \\
    x_2 \\
    \vdots \\
    x_n
  \end{array}
\right] =
\left[
  \begin{array}{c}
    b_1 \\
    b_2 \\
    \vdots \\
    b_m
  \end{array}
\right],
\end{equation} quando visto componente a componente, representa:
\begin{equation}
\left\{
  \begin{array}{cc}
  a_{11} x_1 + a_{12} x_2 + \cdots + a_{1n} x_n =  b_1 \\
  a_{21} x_1 + a_{22} x_2 + \cdots + a_{2n} x_n =  b_2 \\
    \vdots \\
  a_{m1} x_1 + a_{m2} x_2 + \cdots + a_{mn} x_n =  b_m
\end{array}
\right.
\end{equation}


\subsection*{Exercícios resolvidos}



	\construirExeresol



\begin{exeresol}
Dado os vetores abaixo, resolva $4\vec{u}$, $10\vec{v}$ e $3\vec{u} + 2\vec{v}$.
\begin{equation}
 \vec{u} =
\left[
  \begin{array}{c}
    3 \\
    2 \\
    2
  \end{array}
\right] \quad e \quad
 \vec{v} =
\left[
  \begin{array}{c}
    -2 \\
    4 \\
    9
  \end{array}
\right]
\end{equation}
\end{exeresol}

\begin{resol}
Utilizando a definição de multiplicação de escalar em vetores obtemos para $4\vec{u}$:
\begin{equation}
 4\vec{u} =
4\left[
  \begin{array}{c}
    3 \\
    2 \\
    2
  \end{array}
\right] =
  \left[
  \begin{array}{c}
    12 \\
    8 \\
    8
  \end{array}
\right]
\end{equation}
e para $10\vec{v}$:
\begin{equation}
 10\vec{v} =
10\left[
  \begin{array}{c}
    -2 \\
    4 \\
    9
  \end{array}
\right] =
  \left[
  \begin{array}{c}
    -20 \\
    40 \\
    90
  \end{array}
\right].
\end{equation}
Por fim, para resolver $3\vec{u} + 2\vec{v}$ primeiro temos que multiplicar os escalares pelos vetores $\vec{u}$ e $\vec{v}$ separadamente e, por último, somamos o resultado de cada um. Fazendo direto, temos o seguinte:
\begin{equation}
 3\vec{u} + 2\vec{v} =
3\left[
  \begin{array}{c}
    3 \\
    2 \\
    2
  \end{array}
\right] +
2\left[
  \begin{array}{c}
    -2 \\
    4 \\
    9
  \end{array}
\right] =
\left[
  \begin{array}{c}
    9 \\
    6 \\
    6
  \end{array}
\right] +
  \left[
  \begin{array}{c}
    -4 \\
    8 \\
    18
  \end{array}
\right] =
\left[
  \begin{array}{c}
    5 \\
    14 \\
    24
  \end{array}
\right].
\end{equation}
\end{resol}

\begin{exeresol}
Escreva o sistema de equações abaixo na sua forma matricial.
\begin{equation}
 \left\{
  \begin{array}{ll}
    2x_1 - 4x_2 + 6x_3 = 10 \\
    -3x_1 + x_2 - x_3 = 12 \\
    2x_1 - 3x_2 + 8x_3 = 5
  \end{array}\right.
\end{equation}
\end{exeresol}
\begin{resol}
Para colocar um sistema de equações na sua forma matricial, temos que representá-lo como um produto de uma matriz onde se encontram os coeficientes das variáveis e um vetor contendo as variáveis. Temos, assim:
\begin{equation}
 \left[
  \begin{array}{ccc}
    2 & -4 & 6  \\
    -3 & 1 & -1  \\
    2 & -3 & 8
  \end{array}
\right]
\left[
  \begin{array}{c}
   x_1 \\
   x_2 \\
   x_3
  \end{array}
  \right] =
\left[
  \begin{array}{c}
   10 \\
   12 \\
   5
  \end{array}
\right]
\end{equation}
Também podemos dizer que $A$ é a matriz dos coeficientes das variáveis, $\vec{x}$ o vetor das variáveis e $\vec{b}$ o vetor contendo os resultados de cada equação do sistema.
\end{resol}













\begin{exeresol}
Escreva o vetor $\vec{v}$ como uma combinação linear dos vetores $\vec{e_1}$, $\vec{e_2}$, e $\vec{e_3}$.
\begin{equation}
 \vec{v} =
\left[
  \begin{array}{c}
  1 \\
  -2 \\
  3
 \end{array}
\right] \quad , \quad
 \vec{e_1} =
  \left[
  \begin{array}{c}
  1 \\
  1 \\
  1
 \end{array}
\right] \quad , \quad
 \vec{e_2} =
  \left[
  \begin{array}{c}
  1 \\
  2 \\
  4
 \end{array}
\right] \quad , \quad
 \vec{e_3} =
  \left[
  \begin{array}{c}
  2 \\
  -1 \\
  3
 \end{array}
\right]
\end{equation}
\end{exeresol}
\begin{resol}
Podemos representar o vetor $\vec{v}$ como uma combinação linear dos outros vetores da seguinte maneira:
\begin{equation}
 \vec{v} = c_1\vec{e_1} + c_2\vec{e_2} + c_3\vec{e_3}
\end{equation}
Assim, temos:
\begin{equation}
\vec{v} =
\left[
  \begin{array}{c}
  1 \\
  -2 \\
  3
 \end{array}
\right] =
c_1\left[
 \begin{array}{c}
  1 \\
  1 \\
  1
 \end{array}
\right] +
c_2\left[
 \begin{array}{c}
  1 \\
  2 \\
  4
 \end{array}
\right] +
c_3\left[
  \begin{array}{c}
  2 \\
  -1 \\
  3
 \end{array}
\right]
\end{equation}
Usando as propriedades anteriormente estudadas, obtemos:
\begin{equation}
 \left[
  \begin{array}{c}
  1 \\
  -2 \\
  3
 \end{array}
\right] =
\left[
 \begin{array}{c}
  c_1 \\
  c_1 \\
  c_1
 \end{array}
\right] +
\left[
 \begin{array}{c}
  c_2 \\
  2c_2 \\
  4c_2
 \end{array}
\right] +
\left[
  \begin{array}{c}
  2c_3 \\
  -c_3 \\
  3c_3
 \end{array}
\right] =
\left[
  \begin{array}{c}
   c_1 + c_2 + 2c_3 \\
   c_1 + 2c_2 - c_3 \\
   c_1 + 4c_2 + 3c_3
  \end{array}
\right]
\end{equation}
Observamos com esse último resultado que temos um sistema de equações lineares:
\begin{equation}
\left[
  \begin{array}{c}
   c_1 + c_2 + 2c_3 \\
   c_1 + 2c_2 - c_3 \\
   c_1 + 4c_2 + 3c_3
  \end{array}
\right] =
\left[
  \begin{array}{c}
   1 \\
   -2 \\
   3
  \end{array}
\right]
\end{equation}
Usando o método já estudado para resolver sistemas lineares, achamos os valores para $c_1$, $c_2$ e $c_3$. Verifique que $c_1 = -\frac{3}{2}$, $c_2 = \frac{3}{10}$ e $c_3 = \frac{11}{10}$. Esses valores são os coeficientes que fazem com que o vetor $\vec{v}$ possa ser escrito como uma combinação linear dos demais vetores. Assim, obtemos como  resposta:
\begin{equation}
 \vec{v} = -\frac{3}{2}\vec{e_1} + \frac{3}{10}\vec{e_2} + \frac{11}{10}\vec{e_3} \lhd
\end{equation}
\end{resol}

\begin{exeresol}
 Para qual valor de $k$ o vetor $\vec{v}$ é uma combinação linear dos vetores $\vec{e_1}$ e $\vec{e_2}$ ?
\begin{equation}
  \vec{v} =
\left[
  \begin{array}{c}
  1 \\
  -2 \\
  k
 \end{array}
\right] \quad , \quad
 \vec{e_1} =
  \left[
  \begin{array}{c}
  3 \\
  0 \\
  -2
 \end{array}
\right] \quad , \quad
 \vec{e_2} =
  \left[
  \begin{array}{c}
  3 \\
  -2 \\
  5
 \end{array}
\right]
\end{equation}
\end{exeresol}
\begin{resol}
 Escrevemos o vetor $\vec{v}$ como uma combinação linear dos outros vetores da seguinte maneira:
\begin{equation}
\vec{v} = c_1\vec{e_1} + c_2\vec{e_2}
\end{equation}
Obtemos então:
\begin{equation}
\left[
  \begin{array}{c}
  1 \\
  -2 \\
  k
 \end{array}
\right] =
c_1\left[
  \begin{array}{c}
  3 \\
  0 \\
  -2
 \end{array}
\right] +
c_2\left[
  \begin{array}{c}
  3 \\
  -2 \\
  5
 \end{array}
\right] =
\left[
  \begin{array}{c}
  3c_1 \\
  0c_1 \\
  -2c_1
 \end{array}
\right] +
\left[
  \begin{array}{c}
  3c_2 \\
  -2c_2 \\
  5c_2
 \end{array}
\right] =
\left[
\begin{array}{c}
 3c_1 + 3c_2 \\
 -2c_2 \\
 -2c_1+5c_2
\end{array}
\right]
\end{equation}
Assim, temos:
\begin{equation}
\left[
\begin{array}{c}
 3c_1 + 3c_2 \\
 -2c_2 \\
 -2c_1+5c_2
\end{array}
\right] =
\left[
 \begin{array}{c}
  1 \\
  -2 \\
  k
 \end{array}
\right]
\end{equation}
Novamente, temos um sistema de equações lineares para resolver. Verifique que $c_1 = -\frac{2}{3}$ e $c_2 = 1$. Por fim, pela equação envolvendo $k$:
\begin{equation}
 k = -2c_1+5c_2 = \frac{4}{3} + 5 = \frac{19}{3}
\end{equation}
Ou seja, o valor de $k$ que faz com que $\vec{v}$ possa ser representado como uma combinação linear de $\vec{e_1}$ e $\vec{e_2}$ é $k = \frac{17}{3}$.
\end{resol}



\subsection*{Exercícios}

\construirExer




\section{Sistemas lineares e combinações lineares das colunas}

A associação feita acima pode ser resumida como:
% \begin{equation}
% \boxed{\begin{eqnarray}
% \text{Resolver o sistema linear $A \vec{x} = \vec{b}$ equivale a decidir se o} \\ \text{vetor $\vec{b}$ é uma combinação linear das colunas de $A$.}
% \end{eqnarray}}
%\end{equation}
\begin{center}
  ''resolver o sistema linear $A \vec{x} = \vec{b}$ equivale a decidir
  se o vetor $\vec{b}$ é uma combinação linear das colunas de $A$''
\end{center}
ou ainda
\begin{center}
  ''resolver o sistema linear $A \vec{x} = \vec{b}$ equivale a decidir se o vetor $\vec{b}$ pertence ao espaço gerado pelas colunas de $A$''
\end{center}
% \begin{equation}
% \boxed{\begin{split}
% \text{Resolver o sistema linear $A \vec{x} = \vec{b}$ equivale a decidir se o} \\ \text{vetor $\vec{b}$ pertence ao espaço gerado pelas colunas de $A$.}
% \end{split}}
% \end{equation}

Desta forma, tudo o que já vimos sobre existência de soluções para sistemas lineares pode ser ``traduzido'' para o contexto de combinações lineares e espaços gerados (faça as traduções!).


\begin{ex}
Considere o sistema linear do Exemplo \ref{example1}:
\begin{equation}
\left\{
\begin{array}{ll}
x_1 + 2x_2 + x_3 = 12 \\
x_1 -3x_2 + 5x_3 = 1 \\
2x_1 - x_2 + 3x_3 = 10
\end{array}
\right.
\end{equation}
\begin{enumerate}[$(i)$]
\item Vimos que este sistema pode ser representado em \textbf{forma matricial} por
\begin{equation}
\left[
\begin{array}{ccc}
1 &  2 & 1  \\
1 & -3 & 5  \\
2 & -1 & 3
\end{array}
\right]
\left[
\begin{array}{c}
x_1   \\
x_2  \\
x_3
\end{array}
\right] =
\left[
\begin{array}{c}
12   \\
1  \\
10
\end{array}
\right].
\end{equation} Este formato é interessante principalmente para uma análise mais teórica. Podemos escalonar a matriz para decidir sobre a existência de soluções. Podemos também encarar o sistema como uma equação \textit{aparentemente} escalar $A \vec{x} = \vec{b}$; por exemplo, mais adiante no nosso curso, vamos tentar entender quando que pode fazer sentido escrever a solução como $\vec{x} = A^{-1} \vec{b}$; isto é o que se faria em uma equação linear escalar (adiantando: nem sempre se pode fazer isso!).
\item Também podemos representar o sistema como uma \textbf{matriz aumentada associada}:
\begin{equation}
\left[
\begin{array}{ccc|c}
1 &  2 & 1 & 12 \\
1 & -3 & 5 &  1 \\
2 & -1 & 3 & 10
\end{array}
\right].
\end{equation} Esta notação é boa para a resolução do sistema. Escalonando a matriz aumentada associada, conseguiremos dizer se o sistema não possui solução (no caso em que uma linha da forma escalonada for contraditória), se o sistema possui apenas uma solução (quando todas as colunas referentes às variáveis do sistema possuem posição de pivô) ou então se existem infinitas soluções (caso sem linhas contraditórias em que existem colunas sem posição de pivô -- associadas com variáveis livres).
\item Outra forma possível é a \textbf{forma vetorial} (equivalentemente \textbf{combinação linear das colunas} da matriz associada)
\begin{equation}
x_1
\left[
\begin{array}{c}
1   \\
1  \\
2
\end{array}
\right] + x_2
\left[
\begin{array}{c}
  2   \\
 -3   \\
 -1
\end{array}
\right] + x_3
\left[
\begin{array}{c}
 1  \\
 5  \\
 3
\end{array}
\right] =
\left[
\begin{array}{c}
12   \\
1  \\
10
\end{array}
\right].
\end{equation}

Esta maneira de representar o sistema é favorável para descrições geométricas, pois já vimos como interpretar geometricamente combinações lineares de vetores. Também será interessante futuramente no curso quando a imagem de transformações lineares.
\end{enumerate}  O entendimento dessas interpretações diferentes facilita (e muito!) o entendimento de vários conceitos que veremos mais adiante no nosso curso de Álgebra Linear.
% É importante que saibamos passar de uma representação para outra sem dificuldades.
\end{ex}

\subsection*{Exercícios resolvidos}

\construirExeresol

\subsection*{Exercícios}

\construirExer

\section{Exercícios finais}

\construirExer

%\end{document}
